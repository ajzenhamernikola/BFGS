\documentclass[serbian]{beamer}

\usepackage{babel}
\mode<presentation>
{
  \usetheme{default}      % or try Boadilla, Rochester, ...
  \usecolortheme{default} % or try beaver, dove, ...
  \usefonttheme{default}  % or try serif, structurebold, ...
  \setbeamertemplate{navigation symbols}{}
  \setbeamertemplate{caption}[numbered]
} 

% This is for slide numbering
\setbeamertemplate{sidebar right}{}
\setbeamertemplate{footline}{%
	\hfill\usebeamertemplate***{navigation symbols}
	\large\insertframenumber{}/\inserttotalframenumber\hspace{0.5cm}\vspace{0.2cm}
}

\usepackage{listings}
\lstset{
	basicstyle=\ttfamily\small,
	columns=fullflexible,
	keepspaces=true,
	showstringspaces=false,
	escapechar=\%,
	tabsize=4
}

\usepackage{amsmath}
\newcommand\numberthis{\addtocounter{equation}{1}\tag{\theequation}}

\title[BFGS]{BFGS}
\author{Nikola Ajzenhamer}
\institute{Matemati\v cki fakultet, \\ Univerzitet u Beogradu}
\date{\today.}

\begin{document}

\begin{frame}
  	\titlepage
\end{frame}

\begin{frame}{Sadr\v zaj}
  	\tableofcontents
\end{frame}

\section{Uvod}

\begin{frame}{Uvod}

\begin{itemize}
	\item Broyden-Fletcher-Goldfarb-Shanno
	\item Jedan od Kvazi-Njutnovih metoda:
	\begin{itemize}
		\item Zahtevaju gradijent ciljne funkcije pri svakoj iteraciji
		\item Konstrui\v su model ciljne funkcije \\
		koji je dovoljno dobar da daje superlinearnu konvergenciju
		\item S obzirom da ne zahtevaju izvode drugog reda, \\
		ponekad su br\v zi od Njutnovih metoda
	\end{itemize}
\end{itemize}

\end{frame}

\section{BFGS metod}

\begin{frame}{BFGS metod}

\begin{itemize}
	\item Model ciljne funkcije:
	\begin{equation}\label{for:model ciljne funkcije}
		m_k(p) = f_k + \nabla f_k^Tp + \dfrac{1}{2} p^T B_k p
	\end{equation}

	\item $B_k$ je:
	\begin{itemize}
		\item Dimetri\v cna pozitivno definitna matrica dimenzija $n \times n$
		\item A\v zurira se pri svakoj iteraciji
		\item Predstavlja aproksimaciju Hesijana
	\end{itemize}
\end{itemize}

\end{frame}

\begin{frame}{BFGS metod (2)}

\begin{itemize}
	\item Minimizator modela \ref{for:model ciljne funkcije}:
	\begin{equation}\label{for:pravac pretrage}
		p_k = -B_k^{-1} \nabla f_k
	\end{equation}
	
	\item Koristi se kao pravac pretrage:
	\begin{equation*}
		x_{k+1} = x_k + \alpha_k p_k
	\end{equation*}
	
	\item Kasnije \'cemo razmotriti odre\dj ivanje koraka $\alpha_k$
\end{itemize}

\end{frame}

\begin{frame}{BFGS metod (3)}

\begin{itemize}
	\item Mo\v ze se izvesti naredna formula \\
	koja se koristi za izra\v cunavanje naredne aproksimacije $B_{k+1}^{-1}$ \\
	ukoliko je poznata prethodna aproksimacija $B_k^{-1}$:
	\begin{equation}\label{for:BFGS formula}
		B_{k+1}^{-1} = (I - \rho_k s_k y_k^T)B_k^{-1}(I - \rho_k y_k s_k^T) + \rho_k s_k s_k^T
	\end{equation}
	gde va\v zi
	\begin{equation*}
		s_k = x_{k+1} - x_k, \quad y_k = \nabla f_{k+1} - \nabla f_k, \quad \rho_k = \dfrac{1}{y_k^T s_k}
	\end{equation*}
	
	\item Za po\v cetnu aproksimaciju mo\v ze se uzeti $B_0^{-1} = I$
\end{itemize}

\end{frame}

\begin{frame}{Odre\dj ivanje koraka $\alpha_k$}

\begin{itemize}
	\item Odre\dj ivanje $\alpha_k$ se vr\v si tako da zadovoljava Wolfe uslove:
	\begin{equation}\label{for:Wolfe uslovi}
	\begin{aligned}
		f(x_k + \alpha_k p_k) &\leq f(x_k) + c_1 \alpha_k \nabla f_k^T p_k \\
		\nabla f(x_k + \alpha_k p_k)^T p_k &\geq c_2 \nabla f_k^T p_k 
	\end{aligned}
	\end{equation}
	gde va\v zi $0 < c_1 < c_2 < 1$.
	
	\item Prvi uslov govori da $\alpha_k$ treba da daje \emph{neophodan spust} \\ 
	ciljne funkcije $f$ (\emph{Armijo condition})
	
	\item Drugi uslov elimini\v se premale korake (\emph{curvature condition})
\end{itemize}

\end{frame}

\section{Algoritmi}

\begin{frame}{Odre\dj ivanje koraka $\alpha_k$}

\begin{itemize}
	\item Algoritam linijske pretrage za odre\dj ivanje koraka \\
	vr\v si se u dve faze:
	\begin{enumerate}
		\item Faza odsecanja -- pronalazi se interval koji sadr\v zi prihvatljive vrednosti
		\item Faza selekcije -- pronalazi se finalni korak (algoritam \texttt{Zoom})
	\end{enumerate}
\end{itemize}

\end{frame}

\begin{frame}{Algoritam \texttt{Linijska\_pretraga}}

\begin{itemize}
	\item Ulaz: 
	\begin{itemize}
		\item Funkcija $\phi(\alpha) = f(x_k + \alpha p_k)$
	\end{itemize}
	\item Izlaz: 
	\begin{itemize}
		\item Parametar $\alpha^*$ koji zadovoljava Wolfe uslove \ref{for:Wolfe uslovi}
	\end{itemize}
\end{itemize}

\end{frame}

\begin{frame}[fragile]{Algoritam \texttt{Linijska\_pretraga}}

\begin{lstlisting}
Postavi %$\alpha_0 \gets 0$%, odaberi %$\alpha_1 > 0$% i %$\alpha_{max}$%;
%$i \gets 1$%;
repeat
	Izra%\texttt{\v c}%unaj %$\phi(\alpha_i)$%;
	if %$\phi(\alpha_i) > \phi(0) + c_1\alpha_i\phi'(\alpha_0)$% or [%$\phi(\alpha_i) \geq \phi(\alpha_{i-1})$% and %$i > 1$%]
		Postavi %$\alpha^* \gets$% Zoom(%$\alpha_{i-1}, \alpha_i$%); Zaustavi;
	Izra%\texttt{\v c}%unaj %$\phi'(\alpha_i)$%;
	if %$|\phi'(\alpha_i)| \leq -c_2 \phi'(0)$%
		Postavi %$\alpha^* \gets \alpha_i$%; Zaustavi;
	if %$\phi'(\alpha_i) \geq 0$%
		Postavi %$\alpha^* \gets$% Zoom(%$\alpha_i, \alpha_{i-1}$%); Zaustavi;
	Izaberi %$\alpha_{i+1} \in (\alpha_i, \alpha_{max})$%;
	%$i \gets i + 1$%;
end (repeat)
\end{lstlisting}

\end{frame}

\begin{frame}{Algoritam \texttt{Zoom}}

\begin{itemize}
	\item Ulaz: 
	\begin{itemize}
		\item Funkcija $\phi(\alpha) = f(x_k + \alpha p_k)$
		\item Vrednosti $\alpha_{lo}$ i $\alpha_{hi}$
	\end{itemize}
	\item Izlaz: 
	\begin{itemize}
		\item Finalni korak $\alpha^*$ iz intervala $[\alpha_{lo}, \alpha_{hi}]$
	\end{itemize}
\end{itemize}

\end{frame}

\begin{frame}[fragile]{Algoritam \texttt{Zoom}}

\begin{lstlisting}
repeat
	Prona%\texttt{\dj}%i probni korak %$\alpha_j$% izme%\texttt{\dj}%u %$\alpha_{lo}$% i %$\alpha_{hi}$%;
	Izra%\texttt{\v c}%unaj %$\phi(\alpha_j)$%;
	if %$\phi(\alpha_j) > \phi(0) + c_1\alpha_j\phi'(\alpha_0)$% or %$\phi(\alpha_j) \geq \phi(\alpha_{lo})$%
		Postavi %$\alpha_{hi} \gets \alpha_j$%;
	else
		Izra%\texttt{\v c}%unaj %$\phi'(\alpha_j)$%;
		if %$|\phi'(\alpha_j)| \leq -c_2 \phi'(0)$%
			Postavi %$\alpha^* \gets \alpha_j$%; Zaustavi;
		if %$\phi'(\alpha_j)(\alpha_{hi} - \alpha_{lo}) \geq 0$%
			Postavi %$\alpha_{hi} \gets \alpha_{lo}$%;
		Postavi %$\alpha_{lo} \gets \alpha_{j}$%;
end (repeat)
\end{lstlisting}

\end{frame}

\begin{frame}{Algoritam \texttt{BFGS}}

\begin{itemize}
	\item Ulaz: 
	\begin{itemize}
		\item Ciljna funkcija $f$ 
		\item Gradijent funkcije $\nabla f$
		\item Po\v cetna ta\v cka $x_0$
		\item Parametar konvergencije $\epsilon$
		\item Inverz aproksimacije Hesijana $B_0^{-1}$
	\end{itemize}
	\item Izlaz: 
	\begin{itemize}
		\item Aproksimacija minimuma $x^*$
	\end{itemize}
\end{itemize}

\end{frame}

\begin{frame}[fragile]{Algoritam \texttt{BFGS}}

\begin{lstlisting}
%$k \gets 0$%;
while %$||\nabla f_k|| > \epsilon$%
	Izra%\texttt{\v c}%unaj pravac pretrage %$p_k$% prema formuli %\ref{for:pravac pretrage}%;
	Postavi %$\alpha_k \gets$% Linijska_pretraga(%$f$%, %$x_k$%, %$p_k$%);
	Postavi %$x_{k+1} \gets x_k + \alpha_k p_k$%;
	Postavi %$s_k \gets x_{k+1} - x_k$%;
	Postavi %$y_k \gets \nabla f_{k+1} - \nabla f_k$%;
	Izra%\texttt{\v c}%unaj %$B_{k+1}^{-1}$% prema formuli %\ref{for:BFGS formula}%;
	%$k \gets k + 1$%;
end (repeat)
\end{lstlisting}

\end{frame}

\begin{frame}{Ocena slo\v zenosti}

\begin{itemize}
	\item Svaka iteracija se izvodi po cenu od $O(n^2)$ \\
	uz cenu izra\v cunavanja vrednosti funkcija i gradijenata
	
	\item S obzirom da nema re\v savanja \\
	sistema linearnih jedna\v cina i matrica-matrica operacija, \\
	ne postoje ni operacije slo\v zenosti $O(n^3)$
\end{itemize}

\end{frame}

\begin{frame}

\vspace*{3em}

{\Large Hvala na pa\v znji!} 

\vspace*{1em}

Pitanja?

\end{frame}

\end{document}